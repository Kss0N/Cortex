\documentclass{report}
\usepackage{graphicx}
\title{Cortex Engine}
\author{Kss0N et al.}
\date{\today}

\begin{document}
\pagenumbering{roman}
\maketitle
\tableofcontents
\newpage

\section*{Preface}






\chapter{Introduction}
\pagenumbering{arabic}



\chapter{Variables and Naming Principles}
The naming is a fundamental challenge to programming. 
The following \textit{de facto} industry standard principles are fully endorsed by Cortex:
\begin{itemize}
    \item Descriptive naming
    \item Consise naming
    \item Meaningful names
\end{itemize}
And outside of these, there are a few extra guidelines in order to make naming impactful.
The general rule is that these SHOULD be followed, however if for some good reason the established conventions have to be violated, Cortex expects a commented explaination preceeding the varaible declaration.
If there is a scenario requiring a principle not covered in the guidelines, as soon as problem arises, the guidelines MUST be updated.
Responsibility is given to the person discovering the uncovered scenario and they also get to define the new rule.
Cortex does not impose character limits on variable names, but do advice to keep the follwing principle:
\begin{quote}
    Long variable names indicate a need to refactor.
\end{quote}
The main exception to that rule are naming of unit tests, since they technically constitute variable names in \emph{g-test}.
The rationale to this exception is that test cases can become extremely precisely defined to where vocabulary gets imprecise if too few words.
It is nontheless good to try keep test names not longer than they have to be and thus refactoring names to make them shorter is adviced.
An example of that could be substitute phrase \emph{add to end} with \emph{append}

\section{Casing}
Type names take \texttt{PascalCase}.
\begin{verbatim}
    struct MyStruct {
        CtxUint32 myValue;
        CtxChar* zMyString;
    };
\end{verbatim}
Variable names take \texttt{camelCase}.
Routine names take \texttt{snake\_case}.

\section{prefix and postfix}

\section{Common Types}
\subsection{Basic Types}
The following are the basic types used by Cortex.
\begin{center}
\begin{tabular}{ c | c | c }
    Name        &Size (bits)& Unsigned? \\
    CtxChar\footnote[1]     & 8             & NA        \\ 
    CtxInt8     & 8         & false     \\  
    CtxUint8    & 8         & true      \\

    CtxInt16    & 16        & false     \\ 
    CtxUint16   & 16        & true      \\

    CtxInt32    & 32        & false    \\ 
    CtxUint32   & 32        & true     \\

    CtxInt64    & 64         & false    \\ 
    CtxUint64   & 64         & true     \\

    CtxInt\footnote[2] & 64  & false    \\ 
    CtxUint\footnote[2]& 64  & true     \\

    CtxSize\footnote[3]& 64  & true     \\

    CtxFloat32\footnote[4]& 32 & false \\
    CtxFloat64\footnote[4]& 64 & false \\

\end{tabular}
\end{center}
\footnote[1]{Char represents 8 bit ASCII per se, but used as strings it represent UTF-8}
\footnote[2]{The convention is usually that the (u)int32 type is assumed. 
Cortex is developed roughly 20 years past the 32 to 64 bit computing transition and thus assumes all devices will use 64 bit computing.
It is strongly adviced against using this type for storage, if a 64-bit integer is stored, use the CtxInt64 and CtxUint64 types instead.}
\footnote[3]{CtxSize strictly refers to the number of bytes that a block of memory occupies. If the number of elements in a sequence is the target, use any of the CtxUint types instead}
\footnote[4]{IEEE 754 single and double precission float standards}

\subsection{Nul-terminated strings (zStrings)}
Nul-terminated strings, henceforth refered to as \texttt{zStrings}, (the z- prefix standing for \emph{z}ero), are sequencial constructs of characters (\texttt{CtxChar}) where the final character (the terminant) is the special \texttt{'$\backslash$0'} character which has ASCII code of zero.
The zStrings encode UTF-8.
A variable that references a zString has the hard prefix \texttt{z*} instead of \texttt{p*} because of the special rules applying to zStrings.

Cortex deems the stdlib C string.h library archaic and unsafe.
The system has it's own implementation of the library that pass extra parameters indicating maximal size.
It also provides support for UTF-8, which stdlib does not.

\chapter{Modularization}
The fundamental unit of code within Cortex is the routine, as Cortex is sought to be written within a procedural paradigm.
The routine category can be subdivided into two types:
\begin{itemize}
    \item \emph{procedure} is a routine that mutates state. It may or may not bring a return value. 
    A routine that only updates a global value and nothing else counts as a procedure. 
    Simultaneously a routine that opens a file also belongs here, as opening a file inherrently mutates global state of the machine.
    \item \emph{function} is a routine that does not mutate state. A function is allowed to read state, for example global variables or the content of a file, but not mutate it.
    \item \emph{pure function} is a function that does not even look at state. Instead everything it handles must be parameterized.
\end{itemize}
Cortex aims to maximize the amount of functions and pure functions. 
These modules are inherrently easier to debug and unit test. 
Cortex does however recognize the pragmatism of procedures. 

When an entire translation unit contains routines of only the functional and pure functional kinds, the translation unit is said to be \emph{logical}.
The contrast is when a translation unit contains code that mutates state, the module is said to be \emph{stateful}.
The goal within Cortex is to have 20\% of translation units be stateful and the other 80\% to be logical.

A few rules of thumb to keep in mind in order to reach said goal include:
\begin{itemize}
    \item \emph{When in doubt; parameterize.}
    It is prefered to pass data as parameters rather than the routine reaching out for global state.
    This inherrently eases issues with coupling that arise with global state.
    
\end{itemize}

\section{Debugging}

\chapter{Memory Management}

\section{Principles}
Cortex takes extreme precaution in order to not destroy systems via dynamic allocations

\subsection{Pedantry}

\begin{quote}
    Though shall not make dynamic allocations in vain.
\end{quote}
Cortex holds that all forms of dynamic allocated memory cause problems. 
These problems are side effects that may lead to even larger problems. 
It is insanely simple to miss a single \textit{free()} at the end of a subsection, which may cause enormous problems later on.
Thus all heap allocations are taking place with care. 
The main measurements in Cortex to mitigate these are:
\begin{itemize}
    \item Heaps
    \item Debug Registry
    \item Allocation Classes
\end{itemize}
It should be noted that every single allocation call should be expected to take at least 1000 cycles to complete.
This is because allocations have to ask the operating system and that process has to pass security checks.

\section{Heaps}
All allocations have to be associated with a heap. The heap can be limited or unlimited.


If \texttt{NULL} is passed as Heap argument on procedure calls, the default heap is allocated upon.

\subsection{Implementation Notes}
On Windows and XBox systems the \emph{WinAPI} called \emph{HeapAPI} is used. 
This offers direct operating system support and simplifies logic significantly.
It offers most of the concepts already. 
The API imposes a few restrictions, first each allocation must be 16 byte aligned, 

\section{Debug Registry}
The Debug Registry is a mechanism for registering allocations for debug purposes. 
It is a layer only activated in the \texttt{debug} configuration.
This is otherwise ignored because of the inherrent extra performance penalty to the allocation proceduring.
On exit, the Debug Registry will output a memory dump if there are undeallocated chunks still held onto, implying there to be a memory leak.
If specified, the Debug Registry also offers the ability to dumb out \emph{all} allocations of a certain class or group of allocation classes, including when they were allocated and when they were freed.
Allocations are registered using a linked list. Each entry has the following properties:
\begin{verbatim}
    DebugRegistry-Entry is record {
        file-name : zString,
        line-number : uint,
        timestamp : uint,
        action : enum Action,
        pointer : void pointer,
        size : uint
    }
\end{verbatim}
The file name property is supplied with the \texttt{\_\_FILE\_\_} macro and the line number property is supplied with the \texttt{\_\_LINE\_\_} macro.
Timestamp is queried from the computer's system clock.
The action is dependent on the procedure call and the pointer is the reference variable to the inital address of the allocation block.
The size is the size of the block \emph{after} the action recorded has been carried out.



\section{Allocation Class}






\chapter{Threading}





\end{document}